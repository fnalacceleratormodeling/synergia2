%\documentclass[prstab,twocolumn]{revtex4}
%\documentclass[prstab,onecolumn]{revtex4}

\documentclass[preprint]{revtex4}

\usepackage{graphicx}
\usepackage{epstopdf} 
\DeclareGraphicsExtensions{.eps, .pdf}
%\usepackage{amssymb, amsmath}
\usepackage{caption}
\usepackage[lofdepth,lotdepth]{subfig}


\begin{document}
%\DeclareMathOperator{\sech}{sech}
\newcommand{\sech}{\mathop{\mathrm{sech}}\nolimits}
\newcommand{\csch}{\mathop{\mathrm{csch}}\nolimits}
\def\H{{\bar{H}}}
\def\E{{\bar{E}}}
\def\h{{\bar{h}}}
\def\e{{\bar{e}}}
\def\M{{\cal{M}}}
\def\ce{{\cal{E}}}
\def\ch{{\cal{H}}}
\def\R{{\cal{R}}}
\def\O{{\cal{O}}}
\def\T{{\bf{\cal{T}}}}
\def\t{\theta}
\def\a{\bar{a}}
\def\b{\bar{b}}
\def\z{\bar{z}}
\def\x{\bar{x}}
\def\y{\bar{y}}
\def\bxi{\bar{\xi}}
\def\u{\bar{u}}
\def\v{\bar{v}}
\def\w{\bar{w}}
\def\f{\bar{f}}
\def\g{\bar{g}}
\def\W{\tilde{W}}
\def\P{\bar{\Phi}}
\def\Ax{\bar{A_x}}
\def\Ay{\bar{A_y}}
\def\A{\bar{A}}
\def\p{\bar{\phi}}
\def\dP{\overline{\partial \Phi}}
\def\dAx{{{\overline{\partial A}}_x}}
\def\dAy{{{\overline{\partial A}}_y}}
\def\tx{{t_x}}
\def\ty{{t_y}}
\def\nx{{n_x}}
\def\ny{{n_y}}
\def\Q{{\bar{Q}}}


%\title{Wake fields in Synergia}
%\author{Alexandru Macridin}




\maketitle

\section{Wake field for Synergia}

The wake field effects in Synergia are introduced using the split operator method. Every step, each macroparticle
momentum is kicked according to

\begin{eqnarray}
\label{eq:wz}
\beta c \Delta p_z &=& -qQ W^{||}(z), \\
\label{eq:wx}
\beta c \Delta p_x &=& -qQ \left( W_X(z) X + W_x(z) x \right)\\
\label{eq:wy}
\beta c \Delta p_y &=& -qQ \left( W_Y(z) Y + W_y(z) y \right),
\end{eqnarray}
\noindent where $Q,X,Y$ ($q,x,y$) represent the charge and horizontal and vertical displacements
of the leading (trailing) particle. The wake functions, $W^{||},W_X, W_Y, W_x, W_y$ 
depend only on the distance, $z$, between the leading and the trailing particle. 



The  Eqs.~\ref{eq:wz},\ref{eq:wx} and \ref{eq:wy} are valid for vacuum chambers
with horizontal and vertical mirror symmetry.  For pipes with circular or rectangular symmetries,
the detuning wakes $W_x(z),W_y(z)$ are zero in the ultrarelativistic limit . 






\subsection{Wake field file}

The wake functions $W^{||}(z),W_X(z), W_Y(z), W_x(z), W_y(z)$  in Synergia simulations are read from a file.

\subsubsection{Reading a wake file}
\label{sssec:rwf}

A typical wake file has 6 columns ($z$, $W_X$, $W_x $, $W_Y$, $W_y$ and $W^{||}$) or less if due to the symmetry
some of the wakes are zero. The order of columns is important. {\bf To make sure there is no mistake check the reading order 
corresponding to your {\em wake\_type} parameter in wake\_field.cc.}

Examples: 
\begin{itemize}

\item wake\_type="XLXTYLYTZ", requires 6 columns wake file, in order $z$, $W_X$ (lead particle X transverse , i.e. XL),
$W_x $ (trailing particle X transverse, i.e XT), $W_Y$, (lead particle Y transverse , i.e. YL),
$W_y $ (trailing particle Y transverse, i.e YT) and $W^{||}$ (longitudinal wake, i.e. Z). 

\item wake\_type="XLYLZ" (used for IOTA, see the wake file {\em IOTA\_straight\_rw\_wake.dat} ) requires 3 columns, in order $z$, $W_X=W_Y$ 
(lead particle X transverse equal to lead particle Y transverse , i.e. XLYL) and 
$W^{||}$ (longitudinal wake, i.e. Z). 
\end{itemize}


\subsubsection{Preparing a wake file}

The units of the quantities in the wake file are:

\begin{itemize}

\item $z[m]$.  $z=\beta c t$
\item  $\frac{W_X}{Z_0 L}[m^{-2} s^{-1}]$. The transverse wake   per unit length,  divided
by the vacuum impedance $Z_0 \approx 376.7 \Omega$. Valid for both leading and trailing transverse wakes.

\item $\frac{W^||}{Z_0 L}[m^{-1} s^{-1}]$.   The longitudinal wake per unit length,  divided
by the vacuum impedance $Z_0 \approx 376.7 \Omega$.
\end{itemize}

$z$ in the wake file is  on a quadratic grid, i.e every row corresponds to an integer $i$
such  $z_i=i*|i|*\Delta z+\Delta_0$. $i$ can start from negative values if the wake in front of the source particle is 
considered (at finite $\gamma$). $\Delta z$ determines the grid resolution and $\Delta_0 \ll \Delta z$ is chosen to avoid 
the point $z=0$.

Example:

The wake file for IOTA was calculated as described in {\em A. Macridin, et. al,  FERMILAB-PUB-12-518-CD}
for a circular pipe.

\subsection{Impedance in Synergia}

The {\bf \em Impedance} object in Synergia is a collective operator and should be implemented 
in Synergia in  the same way the space charge solvers are implemented (via the  {\bf \em Stepper class}).

\noindent The constructor is:

\noindent impedance\_op=synergia.collective.Impedance(wake\_file, wake\_type, zgrid, lattice\_length, bucket\_distance, registred\_turns, full\_machine, wave\_number)

\noindent where

\begin{itemize}

\item wake\_file (string) is the name of the wake file  
\item wake\_type (string) is the type of the file, see the discussion  in Sec~\ref{sssec:rwf}. For IOTA, wake\_type="XLYLZ".
\item zgrid (integer) determines the number of slices the bunch is longitudinally divided.
\item lattice\_length (double, float) is the ring length.
\item  bucket\_distance (double, float) the distances between buckets
\item  registred\_turns (integer) The number of previous turns considered to produce wakes.  
\item full\_machine (bool).  When full\_machine=1 one considers that all the buckets are filled with identical bunches.
It is a single bunch calculation (since all the bunches are identical), but the bunch feels the wake 
produced by the other buckets. The multi-bunch instabilities are suppressed in this approximation.
\item wave\_number (integer type vector of size 3). wave\_number=[wnx,wny,wnz]. Relevant when full\_machine=1. The buckets are filled with
identical bunches which are displaced along the ring with the wave number wn. For example, the horizontal displacement of 
the buckets along the ring is modulated by $\cos\left(\frac{2 \pi ~\text{wnx bucket\_index}}{\text{num\_buckets}}\right)$;

\end{itemize}

\end{document}

